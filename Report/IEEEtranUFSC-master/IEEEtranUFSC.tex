%%------------------------------------------------------
% 
%% UNIVERSIDADE FEDERAL DE SANTA CATARINA - UFSC
%
%% Prof.: Wyllian B. da Silva
%%

\documentclass[journal]{IEEEtran}

\documentclass[a4paper,10pt]{article}
\usepackage[utf8]{inputenc}

%opening
\title{SNAKE COLORS}
\author{John Thomas Coelho, Patric Lara Ferrari}

\begin{document}

\maketitle

%%----------------- Título
\title                                                {Snake Colors}

\newcommand{\emailautor}                              {johnthomas.style@hotmail.com}

\newcommand{\siglaRevista}                            {UFSC}

\newcommand{\Revista}                                 {Universidade Federal de Santa Catarina (UFSC)}

\begin{abstract}
%% Abstract
\IEEEtitleabstractindextext{

  {\selectlanguage{brazil}
    \begin{abstract}
    
    \end{abstract}
    %%----------------- Keywords
    \renewcommand\IEEEkeywordsname{Palavras-chave} %% Palavras-chave ao invés de 'Index Terms'
    \begin{IEEEkeywords}
    Baseado no classico jogo de celular Snake (Jogo da cobrinha), decidimos refazer o jogo no Linux/Terminal.
    \end{IEEEkeywords}
  }
%------------------------------------------------------
% Section
\section{Introdução}
Baseado no classico jogo de celular Snake (Jogo da cobrinha), decidimos refazer o jogo no Linux/Terminal.
A lógica do jogo consiste em cada vez que a cobrinha se alimenta ela cresce de tamannho acrescentendo 1 a sua pontuação e mudando de cor e velocidade de acordo com mais pontos pego. 
Foram criados 3 níveis. No primeiro (level 1) é apenas o mapa normal com a pontuação que vai até 10.
No level 2 obstaculos são adicionaos a cada pontuação feita, indo até 20 pontos.
Nível 3, a cada 5 movimentos o ponto ira mudar de lugar, somando isso ao obstaculo ja istalados, o nivel 3 apresenta grande dificuldade.
%------------------------------------------------------

%%subsection
\subsection{Bibliotecas}
<termios.h> 
<unstd.h>
<fcntl.h>
São utilizadas por conta da função int kbhit(void). É uma função inexistente no Linux, mas era necessario para fazer o jogo rodar.
Sua função é ler o teclado a todo tempo e enquanto nenhuma tecla é pressionada ela retorna 0 par ao computador, e quanto é pressionada retorna 1.
<n.curses.h>
Manipulação do terminal.
Para fazer o download da Biblioteca ncurses:
sudo apt-get install libncurses5-dev libncursesw5-dev.
<time.h>
Manipula o tempo.

%
%-----------------------------------------------------
%
\subsection{Define}
Na parte de definição foi utilizado cordenadas, são elas: Horizontal, vertical, cima, baixo, esquerda e direita. São os sentidos que a snake toma.
Para movimentar a snake no terminal são usados as teclas W,A,S,D.

O int global_move=0, é um contador que conta quantos movimentos foi feito. É usado no terceiro nível para contar os 5 movimentos.

Typedif Struct Snake
Cordenadas em X e Y para calda e cabeça mais a velocidade.
%
%-----------------------------------------------------
%
\subsection{Main código principal}
Srand (time(NULL)) cria a semente e o tempo faz com que saia em um lugar aleatório.
em seguida temos uma função ``do'', ``while''. Entrada e saida do jogo.
Dentro do ``do'' é imposto algumas declarações como o proprio snake, snake1, velocidade, posição e comida.
O snake_init é um procedimento que define as cordenadas da calda em X e Y e as cordenadas da cabeça também em X e Y. (codenada da calda mais o tamanho menos um.

Food_init criação aleatória para a comida dentro dos limites definido no terminal.
Void snake_place procedimento que define as posições da snake.
Set_borders printa e cria os limites da tela. Criação dada através de laços em X e Y criando as paredes
Horizontal e vertical.
%
%-----------------------------------------------------
%
\subsection{Movimento}
Move_head (snake *snake1, snake_pos *pos1)
Controle pelos direcionais. Quando é pressionado A, S, W, D entra no laço que soma ou subtrai as cordenadas.
O ``if'' a baixo não pega nenhum ponto e começa o procedimento com o move_tail, onde a calda passa ele printa um espaço  em branco.
No ``for'' atualiza a posição do corpo da snake..

else quando a snake se alimenta, aumenta o tamanho e a velcidade e acrescenta 1 a sua pontuação.

Obstaculo_init cria um obstaculo aleatório nos limetes do jogo.
obstaculo_print, printa os obstaculos (8).
%
%-----------------------------------------------------
\subsection{Game Over}
Colidindo com a parede, corpo ou obstaculos (8), é disparado um som de fim de jogo e é dado a opção se o usuario quer ternornar ao jogo.
%%------------------------------------------------------
%% Section
\section{Conclusão}
O objetivo final para a conclusão na materia de progração 1 ministrada pelo professor Wyllian B. da Silva, foi a criação de um jogo usando os recursos do Linux, e da linguagem c, o executando no terminal. Portando dessas ferramentas, aperfeiçoamos nossa capacidade no desenvolvimento de algoritmos alem da pratica na utilização de matrizes, ponteiros e vetores, bem como na combinação de procedimentos e funções.

\end{document}
