%%------------------------------------------------------
% 
%% UNIVERSIDADE FEDERAL DE SANTA CATARINA - UFSC
%
%% Prof.: Wyllian B. da Silva
%%
%% Template: estilo IEEEtran [paper com duas colunas]
%% Adaptado de: https://ieeeauthorcenter.ieee.org/create-your-ieee-article/use-authoring-tools-and-ieee-article-templates/ieee-article-templates/templates-for-transactions/
%               https://ctan.org/tex-archive/macros/latex/contrib/IEEEtran?lang=en

%% Instruções: http://mirrors.ctan.org/macros/latex/contrib/IEEEtran/IEEEtran_HOWTO.pdf
%%
%% Recomendações:
%% Utilize o Edito Kile (SO Linux)
%% Certifique-se de que a codificação de caracteres utilizada é a UTF-8





\documentclass[journal]{IEEEtran}


%%------------------------------------------------------
%% Packages
%%------------------------------------------------------
\usepackage[T1]{fontenc}           %% Codificação de caracteres
\usepackage[utf8]{inputenc}        %% Codificação de caracteres (conversão automática dos acentos)
\usepackage[dvips]{graphicx}       %% para a macro includegraphics 
\usepackage[english,brazil]{babel} %% PT_BR e EN (o último define a prioridade no arquivo)
\usepackage{pgf}                   %% macro para criar gráficos
\usepackage{epsfig}                %% or use the epsfig package if you prefer to use the old commands
\usepackage{graphics}              %% use the graphics package for simple commands
\usepackage{graphicx}              %% or use the graphicx package for more complicated commands
\usepackage{epstopdf}              %% enable EPS (convert to PDF)
\usepackage{float}                 %% float environment
\usepackage{eqparbox}              %% to define a group of boxes 
\usepackage{hyphenat}              %% prevent hyphenation
\usepackage{hyperref}              %% enalbe one-click link

% \usepackage{showframe} %% just for the example

% \usepackage[sort,compress]{cite} %% disable if natbib package is activated
\usepackage[numbers,sort&compress,square]{natbib} %% e.g., [2-5]



%%------------------------------------------------------
%% Definitions
%%------------------------------------------------------

\hyphenation{op-tical net-works semi-conduc-tor}

%% Can use something like this to put references on a page
%% by themselves when using endfloat and the captionsoff option.
\ifCLASSOPTIONcaptionsoff
  \newpage
\fi


%%----------------- Definindo as variáveis com números
\makeatletter
%
\newcommand{\prenome}{\afterassignment\prenome@aux\count0=}
\newcommand{\prenome@aux}{\csname prenome\the\count0\endcsname}
%
\newcommand{\nomedomeio}{\afterassignment\nomedomeio@aux\count0=}
\newcommand{\nomedomeio@aux}{\csname nomedomeio\the\count0\endcsname}
%
\newcommand{\sobrenome}{\afterassignment\sobrenome@aux\count0=}
\newcommand{\sobrenome@aux}{\csname sobrenome\the\count0\endcsname}
\makeatother
%%%%%

%%----------------- Configurações de hyperlinks
%% Não decorado, sem destaque
\hypersetup{
  colorlinks=false,
  pdfborder={0 0 0},
}




%%------------------------------------------------------
%% Configurações
%%------------------------------------------------------

%%----------------- Título
\title {Snake Colors}

\newcommand{\emailautor}                              {seuemail@grad.ufsc.br}

\newcommand{\siglaRevista}{UFSC}

\newcommand{\Revista}{Universidade Federal de Santa Catarina (UFSC)}

% %%----------------- Vários Autores
\author{\authorblockN{\prenomePrincipal\nomedomeioPrincipal\sobrenomePrincipal~\authorrefmark{John Thomas Coelho} e
\prenome2\sobrenome2\authorrefmark{Patric Lara Ferrari} \membership
% 
\\\IEEEauthorblockA{\IEEEauthorrefmark{1}Universidade Federal de Santa Catarina (UFSC)}
% 
}}

%%------------------------------------------------------
%% Abstract
\IEEEtitleabstractindextext{

  {\selectlanguage{brazil}
    \begin{abstract}
     Este relatório tem como objetivo apresentar a formação de um jogo em terminal utilizando a linguagem C de programação. Nele estão contidos os problemas encontrados para execução e os métodos utilizados para que fosse possível formar uma adaptação de Snake Colors.
    \end{abstract}
    %%----------------- Keywords
    \renewcommand\IEEEkeywordsname{Palavras-chave} 
   \begin{IEEEkeywords}
    Programação, linguagem C.
    \end{IEEEkeywords}
  }

  {\selectlanguage{english}
    \begin{abstract}
    This report aims to present the formation of a game in terminal using the C programming language. In it are contained the problems encountered for execution and the methods used so that it was possible to form an adaptation of Snake Colors. 
    \end{abstract}
    %%----------------- Keywords
    \begin{IEEEkeywords}
    Programming, C language, Arcade game.
    \end{IEEEkeywords}
  }
}


\begin{document}



%%%%%%%%%%%%%%%%%%%%%%%%%%%% Inserção de informações
\maketitle
\IEEEdisplaynontitleabstractindextext
\IEEEpeerreviewmaketitle


%%%%%%%%%%%%%%%%%%%%%%%%%%% SECTION INTRODUÇÃO
\section{Introdução}

\IEEEPARstart{E}{ste} Baseado no classico jogo de celular Snake (Jogo da cobrinha), decidimos refazer o jogo no Linux/Terminal.
A lógica do jogo consiste em cada vez que a cobrinha se alimenta ela cresce de tamannho acrescentendo 1 a sua pontuação e mudando de cor e velocidade de acordo com mais pontos pego. 
Foram criados 3 níveis. No primeiro (level 1) é apenas o mapa normal com a pontuação que vai até 10.
No level 2 obstaculos são adicionaos a cada pontuação feita, indo até 20 pontos.
Nível 3, a cada 5 movimentos o ponto ira mudar de lugar, somando isso ao obstaculo ja istalados, o nivel 3 apresenta grande dificuldade.

\end{document}

%%%%%%%%%%%%%%%%%%%%%%%%%%%%%%%%%%%%% Subsection
\subsection{Bibliotecas}
<termios.h> 
<unstd.h>
<fcntl.h>
São utilizadas por conta da função int kbhit(void). É uma função inexistente no Linux, mas era necessario para fazer o jogo rodar.
Sua função é ler o teclado a todo tempo e enquanto nenhuma tecla é pressionada ela retorna 0 par ao computador, e quanto é pressionada retorna 1.
<n.curses.h>
Manipulação do terminal.
Para fazer o download da Biblioteca ncurses:
sudo apt-get install libncurses5-dev libncursesw5-dev.
<time.h>
Manipula o tempo.

%%%%%%%%%%%%%%%%%%%%%%%%%%%%%%%%%%%% Subsubsection

\subsection{Define}
Na parte de definição foi utilizado cordenadas, são elas: Horizontal, vertical, cima, baixo, esquerda e direita. São os sentidos que a snake toma.
Para movimentar a snake no terminal são usados as teclas W,A,S,D.

O int global_move=0, é um contador que conta quantos movimentos foi feito. É usado no terceiro nível para contar os 5 movimentos.

Typedif Struct Snake
Cordenadas em X e Y para calda e cabeça mais a velocidade.

%%%%%%%%%%%%%%%%%%%%%%%%%%%%%%%%%%%% Subsubsection
\subsection{Main código principal}
Srand (time(NULL)) cria a semente e o tempo faz com que saia em um lugar aleatório.
em seguida temos uma função ``do'', ``while''. Entrada e saida do jogo.
Dentro do ``do'' é imposto algumas declarações como o proprio snake, snake1, velocidade, posição e comida.
O snake_init é um procedimento que define as cordenadas da calda em X e Y e as cordenadas da cabeça também em X e Y. (codenada da calda mais o tamanho menos um.

Food_init criação aleatória para a comida dentro dos limites definido no terminal.
Void snake_place procedimento que define as posições da snake.
Set_borders printa e cria os limites da tela. Criação dada através de laços em X e Y criando as paredes
Horizontal e vertical.

%%%%%%%%%%%%%%%%%%%%%%%%%%%%%%%%%%%% Subsubsection
\subsection{Movimento}
Move_head (snake *snake1, snake_pos *pos1)
Controle pelos direcionais. Quando é pressionado A, S, W, D entra no laço que soma ou subtrai as cordenadas.
O ``if'' a baixo não pega nenhum ponto e começa o procedimento com o move_tail, onde a calda passa ele printa um espaço  em branco.
No ``for'' atualiza a posição do corpo da snake..

else quando a snake se alimenta, aumenta o tamanho e a velcidade e acrescenta 1 a sua pontuação.

Obstaculo_init cria um obstaculo aleatório nos limetes do jogo.
obstaculo_print, printa os obstaculos (8).

%%%%%%%%%%%%%%%%%%%%%%%%%%%%%%%%%%%% Subsubsection
\subsection{Game Over}
Colidindo com a parede, corpo ou obstaculos (8), é disparado um som de fim de jogo e é dado a opção se o usuario quer ternornar ao jogo.

%%%%%%%%%%%%%%%%%%%%%%%%%%%%%%%%%%%% Section
\section{Conclusão}
O objetivo final para a conclusão na materia de progração 1 ministrada pelo professor Wyllian B. da Silva, foi a criação de um jogo usando os recursos do Linux, e da linguagem c, o executando no terminal. Portando dessas ferramentas, aperfeiçoamos nossa capacidade no desenvolvimento de algoritmos alem da pratica na utilização de matrizes, ponteiros e vetores, bem como na combinação de procedimentos e funções.  


%%%%%%%%%%%%%%%%%%%%%%%%%%% References (Option 2): incorporeted
 \begin{thebibliography}{10} 
 
   \bibitem{Pinheiro:2012} 
     Pinheiro, F. A. C., 2012. {\em Elementos de programação em C.} Porto Alegre: Bookman, 2012. 
     978-85-407-0202-8.

   \bibitem{da Silva:2019} 
     da Silva, W. B. 2019. In: {\em CPF}, Curitiba. <https://github.com/wyllianbs/CPF>.
 
 \end{thebibliography}



\end{document}
